%\documentclass[11pt,a4paper]{scrbook}
\documentclass[12pt,a4paper,listof=totoc,captions=nooneline]{scrreprt} 
\usepackage{geometry}	
\usepackage[utf8]{inputenc}
\usepackage[T1]{fontenc}
\usepackage[pdftex]{graphicx}
\usepackage[ngerman]{babel}
\usepackage{colortbl}	
\usepackage{xcolor}
\usepackage{soul}
\usepackage{float}
\usepackage{caption}
\usepackage{subcaption}
%\usepackage{tocloft}
\usepackage[colorlinks,
pdfpagelabels,
pdfstartview = FitH,
bookmarksopen = true,
bookmarksnumbered = true,
linkcolor = black,
plainpages = false,
hypertexnames = false,
citecolor = black] {hyperref}
\usepackage{amsmath, amssymb, amstext, amsfonts, mathrsfs}	% Mathe
\usepackage{selinput} 
\SelectInputMappings{ 
adieresis={�}, 
germandbls={�}, 
}
\renewcommand{\familydefault}{\sfdefault}
\usepackage{graphicx}
\usepackage[onehalfspacing]{setspace}

\usepackage{chngcntr}
\counterwithout{equation}{chapter}


\definecolor{uhhred}{cmyk}{0,100,100,0}

\counterwithout{figure}{chapter}
\counterwithout{table}{chapter}
\addtokomafont{caption}{\centering}
\AfterTOCHead[lof]{\renewcommand*{\numberline}[1]{\nonumberline}}
\AfterTOCHead[toc]{\renewcommand*{\numberline}[1]{\nonumberline}}

\usepackage{selinput} 
\SelectInputMappings{ 
adieresis={�}, 
germandbls={�}, 
}
\renewcommand{\familydefault}{\sfdefault}
\usepackage{graphicx}
\usepackage[onehalfspacing]{setspace}


\definecolor{uhhred}{cmyk}{0,100,100,0}

\counterwithout{figure}{chapter}
\counterwithout{table}{chapter}
\addtokomafont{caption}{\centering}
\AfterTOCHead[lof]{\renewcommand*{\numberline}[1]{\nonumberline}}
\AfterTOCHead[toc]{\renewcommand*{\numberline}[1]{\nonumberline}}

\begin{document}

%\frontmatter
\newgeometry{centering,left=2cm,right=2cm,top=2cm,bottom=2cm}
\begin{titlepage}
\includegraphics[scale=0.3]{UHH-Logo_2010_Farbe_CMYK.pdf}
\vspace*{2cm}
\Large
\begin{center} 
      {\color{uhhred}\textbf{\so{Labreport, Gruppe 4 }}}
%oder {\color{uhhred}\textbf{\so{MASTERTHESIS}}}
\vspace*{2.0cm}\\
{\LARGE \textbf{Projekt Netzwerk-Infrastruktur WS 2017/18}}
\vspace*{2.0cm}\\
vorgelegt von
\vspace*{0.4cm}\\
\textbf{Dewin Bagci:} 5bagci@informatik.uni-hamburg.de\\ \textbf{Karan Popat:} karan.popat@outlook.de \\ \textbf{ Hanife Demircioglu:} h.demircioglu@hotmail.de \\ 
\end{center}
\vspace*{3.9cm}

\noindent 
MIN-Fakult�t \vspace*{0.4cm} \\ 
Fachbereich Informatik \\
Abgabedatum: 01.03.2018\\
Dozent: Robert Olotu

\end{titlepage}

\restoregeometry

\tableofcontents

\chapter{Abk�rzungsverzeichnis}

\listoffigures
%\mainmatter 


\chapter*{Part 3: Network Troubleshooting Utilities}
\addcontentsline{toc}{chapter}{Part 3: Network Troubleshooting Utilities} 

\begin{figure}[H]
%\begin{subfigure}{0.5\textwidth}

\includegraphics[width=0.75\textwidth]{UHH-Logo_2010_Farbe_CMYK.pdf}
%\subcaption{Subfigure Bild Nr. 1}

%\end{subfigure}
%\begin{subfigure}{0.5\textwidth}
%\includegraphics[width=0.75\textwidth]{Bild2.png}
%\subcaption{Subfigure Bild Nr. 2}
%\end{subfigure}
\caption[Abbildung 1: Broadcast und Multicast]{Broadcast und Multicast}
\end{figure}


\section*{Exercise 6: Managing Services (Please use pnidX-svr-mu)}
\addcontentsline{toc}{section}{Exercise 6: Managing Services (Please use pnidX-svr-mu} 

\section*{Exercise 7: Configure the following network (figure 1) using ifconfig and route add}
\addcontentsline{toc}{section}{Exercise 7: Configure the following network (figure 1) using ifconfig and route add} 

\section*{Exercise 8: Configure the following network (figure 1) using ip and nmcli}
\addcontentsline{toc}{section}{Exercise 8: Configure the following network (figure 1) using ip and nmcli} 

\section*{Exercise 9: Configure the following network (figure 1) using GUI}
\addcontentsline{toc}{section}{Exercise 9: Configure the following network (figure 1) using GUI} 

\chapter*{Part 4: Network Scanning}
\addcontentsline{toc}{chapter}{Part 4: Network Scanning} 

\section*{Exercise 1: Configure the networks of figure 1}
\addcontentsline{toc}{section}{Exercise 1: Configure the networks of figure 1} 

\section*{Exercise 2: NMAP}
\addcontentsline{toc}{section}{Exercise 2: NMAP} 

\section*{Exercise 3: Nessus network device identification}
\addcontentsline{toc}{section}{Exercise 3: Nessus network device identification} 

\section*{Exercise 4: OpenVAS Network device identification}
\addcontentsline{toc}{section}{Exercise 4: OpenVAS Network device identification} 

\chapter*{Part 5: Sniffing, Virtual Private Network (VPN)}
\addcontentsline{toc}{chapter}{Part 5: Sniffing, Virtual Private Network (VPN)} 

\section*{Exercise 1: Configure and set the networks shown below (figure1 and 2)}
\addcontentsline{toc}{section}{Exercise 1: Configure and set the networks shown below (figure1 and 2)} 

\section*{Exercise 2: Getting started with network monitoring tools}
\addcontentsline{toc}{section}{Exercise 2: Getting started with network monitoring tools} 

\section*{Exercise 3: TCPDUMP}
\addcontentsline{toc}{section}{Exercise 3: TCPDUMP} 

\section*{Exercise 4: Wireshark}
\addcontentsline{toc}{section}{Exercise 4: Wireshark} 

\section*{Exercise 5: Experimenting with network monitoring tools}
\addcontentsline{toc}{section}{Exercise 5: Experimenting with network monitoring tools} 

\section*{Exercise 6: Set up a host-to-host VPN using preshared key}
\addcontentsline{toc}{section}{Exercise 6: Set up a host-to-host VPN using preshared key} 

\section*{Exercise 7: Set up a host-to-host VPN using RSA keys}
\addcontentsline{toc}{section}{Exercise 7: Set up a host-to-host VPN using RSA keys}

\section*{Exercise 8: Set up a network-to-network VPN using preshared key}
\addcontentsline{toc}{section}{Exercise 8: Set up a network-to-network VPN using preshared key}

\section*{Exercise 9: Set up a network-to-network VPN using RSA secrets keys}
\addcontentsline{toc}{section}{Exercise 9: Set up a network-to-network VPN using RSA secrets keys}



%\backmatter 

\chapter{Literaturverzeichnis und Quellenverzeichnis}


\end{document}