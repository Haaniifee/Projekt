%\documentclass[11pt,a4paper]{scrbook}
\documentclass[12pt,a4paper,listof=totoc,captions=nooneline]{scrreprt} 
\usepackage{geometry}	
\usepackage[utf8]{inputenc}
\usepackage[T1]{fontenc}
\usepackage[pdftex]{graphicx}
\usepackage{graphicx}
\usepackage[ngerman]{babel}
\usepackage{colortbl}	
\usepackage{xcolor}
\usepackage{soul}
\usepackage{float}
\usepackage{caption}
\usepackage{subcaption}
%\usepackage{tocloft}
\usepackage[colorlinks,
pdfpagelabels,
pdfstartview = FitH,
bookmarksopen = true,
bookmarksnumbered = true,
linkcolor = black,
plainpages = false,
hypertexnames = false,
citecolor = black] {hyperref}
\usepackage{amsmath, amssymb, amstext, amsfonts, mathrsfs}	% Mathe
\usepackage{selinput} 
\SelectInputMappings{ 
adieresis={�}, 
germandbls={�}, 
}
\renewcommand{\familydefault}{\sfdefault}
\usepackage{graphicx}
\usepackage[onehalfspacing]{setspace}

\usepackage{chngcntr}
\counterwithout{equation}{chapter}


\definecolor{uhhred}{cmyk}{0,100,100,0}

\counterwithout{figure}{chapter}
\counterwithout{table}{chapter}
\addtokomafont{caption}{\centering}
\AfterTOCHead[lof]{\renewcommand*{\numberline}[1]{\nonumberline}}
\AfterTOCHead[toc]{\renewcommand*{\numberline}[1]{\nonumberline}}

\usepackage{selinput} 
\SelectInputMappings{ 
adieresis={�}, 
germandbls={�}, 
}
\renewcommand{\familydefault}{\sfdefault}
\usepackage{graphicx}
\usepackage[onehalfspacing]{setspace}


\definecolor{uhhred}{cmyk}{0,100,100,0}

\counterwithout{figure}{chapter}
\counterwithout{table}{chapter}
\addtokomafont{caption}{\centering}
\AfterTOCHead[lof]{\renewcommand*{\numberline}[1]{\nonumberline}}
\AfterTOCHead[toc]{\renewcommand*{\numberline}[1]{\nonumberline}}

\begin{document}
\setlength{\parindent}{0pt}
%\frontmatter
\newgeometry{centering,left=2cm,right=2cm,top=2cm,bottom=2cm}
\begin{titlepage}
\includegraphics[scale=0.3]{UHH-Logo_2010_Farbe_CMYK.pdf}
\vspace*{2cm}
\Large
\begin{center} 
      {\color{uhhred}\textbf{\so{Labreport, Gruppe 4 }}}
%oder {\color{uhhred}\textbf{\so{MASTERTHESIS}}}
\vspace*{2.0cm}\\
{\LARGE \textbf{Projekt Netzwerk-Infrastruktur WS 2017/18}}
\vspace*{2.0cm}\\
vorgelegt von
\vspace*{0.4cm}\\
\textbf{Dewin Bagci:} 5bagci@informatik.uni-hamburg.de\\ \textbf{Karan Popat:} karan.popat@outlook.de \\ \textbf{ Hanife Demircioglu:} h.demircioglu@hotmail.de \\ 
\end{center}
\vspace*{3.9cm}

\noindent 
MIN-Fakult�t \vspace*{0.4cm} \\ 
Fachbereich Informatik \\
Abgabedatum: 01.03.2018\\
Dozent: Robert Olotu

\end{titlepage}

\restoregeometry

\tableofcontents

\chapter{Abk�rzungsverzeichnis}

\listoffigures
%\mainmatter 


\chapter*{Part 3: Network Troubleshooting Utilities}
\addcontentsline{toc}{chapter}{Part 3: Network Troubleshooting Utilities} 

\begin{figure}[H]
%\begin{subfigure}{0.5\textwidth}

\includegraphics[width=0.75\textwidth]{UHH-Logo_2010_Farbe_CMYK.pdf}
%\subcaption{Subfigure Bild Nr. 1}

%\end{subfigure}
%\begin{subfigure}{0.5\textwidth}
%\includegraphics[width=0.75\textwidth]{Bild2.png}
%\subcaption{Subfigure Bild Nr. 2}
%\end{subfigure}
\caption[Abbildung 1: Broadcast und Multicast]{Broadcast und Multicast}
\end{figure}


\section*{Exercise 6: Managing Services (Please use pnidX-svr-mu)}
\addcontentsline{toc}{section}{Exercise 6: Managing Services (Please use pnidX-svr-mu} 

\section*{Exercise 7: Configure the following network (figure 1) using ifconfig and route add}
\addcontentsline{toc}{section}{Exercise 7: Configure the following network (figure 1) using ifconfig and route add} 

\section*{Exercise 8: Configure the following network (figure 1) using ip and nmcli}
\addcontentsline{toc}{section}{Exercise 8: Configure the following network (figure 1) using ip and nmcli} 

\section*{Exercise 9: Configure the following network (figure 1) using GUI}
\addcontentsline{toc}{section}{Exercise 9: Configure the following network (figure 1) using GUI} 

\chapter*{Part 4: Network Scanning}
\addcontentsline{toc}{chapter}{Part 4: Network Scanning} 

\section*{Exercise 1: Configure the networks of figure 1}
\addcontentsline{toc}{section}{Exercise 1: Configure the networks of figure 1} 

\section*{Exercise 2: NMAP}
\addcontentsline{toc}{section}{Exercise 2: NMAP} 

\section*{Exercise 3: Nessus network device identification}
\addcontentsline{toc}{section}{Exercise 3: Nessus network device identification} 

\section*{Exercise 4: OpenVAS Network device identification}
\addcontentsline{toc}{section}{Exercise 4: OpenVAS Network device identification} 

\chapter*{Part 5: Sniffing, Virtual Private Network (VPN)}
\addcontentsline{toc}{chapter}{Part 5: Sniffing, Virtual Private Network (VPN)} 

\section*{Exercise 1: Configure and set the networks shown below (figure1 and 2)}
\addcontentsline{toc}{section}{Exercise 1: Configure and set the networks shown below (figure1 and 2)} 

\section*{Exercise 2: Getting started with network monitoring tools}
\addcontentsline{toc}{section}{Exercise 2: Getting started with network monitoring tools} 

\section*{Exercise 3: TCPDUMP}
\addcontentsline{toc}{section}{Exercise 3: TCPDUMP} 

\section*{Exercise 4: Wireshark}
\addcontentsline{toc}{section}{Exercise 4: Wireshark} 

\textbf{Question 1:} Please type and examine the syntax for a Wireshark command which capture filter so that all IP datagrams with source or destination IP address equal to 10.88.X.? are recorded. \\ 

\textbf{Answer 1:} Mit dem Filter: \textit{ip.addr == 10.88.40.70} k�nnen wir alle Netzwerkpakete, welche �ber die Schnittstelle 10.88.40.70 gesendet oer empfangen werden, abfangen und anzeigen lassen.\\

\begin{figure}[H]
\begin{center}
\includegraphics[width=0.7\textwidth]{bilder/screens/Part_5/Aufgabe_4/Q1.PNG} 
\end{center}
\caption[Abbildung X1 Wireshark Filter 1]{Wireshark Filter ip.addr == 10.88.40.70}
\end{figure}

\textbf{Question 2:} Please type and examine the syntax for a Wireshark display filter that shows IP datagrams with destination IP address equal to 10.88.X.? and frame size greater than 400 bytes. \\

\textbf{Answer 2:} Um alle Datenpakete abzufangen, die mindestens 400 Byte gro� sind, bedarf eine kleine Erweiterung des vorherigen Befehls. Der Filter lautet nun: \textit{ip.addr == 10.88.40.70 \&\& frame.len > 400}. Mit dem Teil \textit{frame.len > X} k�nnen wir die Datenpakete nach Bytegr��e X Filtern. F�r X gilt, X $< 2^{32}$ $\wedge$ X $\in \mathbb{N}$. \\

\textbf{Question 3:} Please type and examine the syntax for a Wireshark display filter that shows packets containing ICMP messages with source or destination IP address equal to 10.88.X.? and frame numbers between 15 and 30 \\

\textbf{Answer 3:} Der Filter lautet: \textit{ip.addr == 10.88.40.70 \&\& (frame.number > 15 \&\& frame.number < 30)}. ICMP steht f�r Internet Control Message Protocol und �bermittelt haupts�chlich Diagnose-informationen zwischen dem Router und dem Host. \\

\textbf{Question 4:} Please type and examine the syntax for a Wireshark display filter that shows packets containing TCP segments with source or destination IP address equal to 10.88.X.? and using port number 23. \\

\textbf{Answer 4:} Damit wir alle TCP Pakete eines Hosts �ber die Port 23 abfangen k�nnen wird der folgende Filter eingesetzt: \textit{ip.dst == 10.88.40.70 and tcp.port == 23}. Bei TCP handelt es sich um ein �bertragungsprotokoll (Transmission Control Protocol) aus der Familie der Internetprotokolle. Port 23 ist standardisiert f�r den Service Telnet. \\

\textbf{Question 5:} Please type and examine a Wireshark capture filter expression for Q4. \\  

\textbf{Answer 5:} Der Filter ist �hnlich wie in Q4, lediglich die Konfiguration findet an einer anderen Stelle statt. \\ 

\textbf{Question 6:} Please type and examine the syntax for a Wireshark command which, by default, collects packets with source or destination IP address 10.88.X.? on interface eth4. \\

\textbf{Answer 6:} Innerhalb des Terminals l�sst sich der Filter: wireshark -i eth4 -k -f "host 10.88.40.70", anwenden. Die Argumente bedeuten dabei folgendes: -i eth4 steht f�r Interface, -k startet das Abfangen von Paketen und -f "host 10.88.40.70", ist der Paketfilter.\\

\textbf{Question 7:} : Please type and examine the syntax of a display filter which selects the TCP packets with destination IP address 10.88.X.?, and TCP port number 23.\\

\textbf{Answer 7:} Der Filter lautet: \textit{ip.addr == 10.88.40.70 \&\& tcp.port == 23} und f�ngt alle ein-/ausgehenden Pakete der Ip Adresse 10.88.40.70 �ber den Port 23 (Telnet) ab.\\

\textbf{Question 8:} Please login to the server pnidX-mid-hh and start an ftp client to the server pnidXcnt-bln(vsftpd
daemon should be running on pnidX-cnt-bln). Please use
wireshark on pnidX-mid-bln to sniff or capture the username and password of the ftp service between pnidX-mid-hh and pnidX-cnt-bln. Is this possible, show your result of the capture\\

\textbf{Answer 8:} Mithilfe von Wireshark k�nnen wir leicht das Telnet login Passwort herausfinden, da bei der �bertragung via Telnet die Pakete unverschl�sselt �bertragen werden. Dazu starten wir zun�chst Wirehsark auf dem Host pnid4-mid-hh. \\

\textbf{Question 9:} Please login to the server pnidX-mid-hh and start an ssh client to the server
pnidX-cnt-bln(sshd daemon should be running on pnidX-cnt-bln). Please use wireshark on pnidX-mid-bln to sniff or capture the username and password of the ssh service between pnidX-mid-hh and pnidX-cnt-bln. Is this possible, show the result of the capture. \\

\textbf{Answer 9:} Anders als Telnet werden bei shh (Secure Shell) die Pakete verschl�sselt �bertragen, sodass es nicht m�glich ist das Passwort mitzulesen. Zuerst starten wir wireshark auf dem Host pnid4-mid-hh. \\

\section*{Exercise 5: Experimenting with network monitoring tools}
\addcontentsline{toc}{section}{Exercise 5: Experimenting with network monitoring tools} 

\textbf{Exercise:} In this exercise you will connect to the webserver pnidX-cnt-bln from pnidX-mid-hh. Let pnidX-cnt-bln determine your IP-address and the OS you are running. Then, connect to
a service of your choice (e.g. ftp, http, ssh etc.) on pnidX cnt-bln. Let pnidX-mid-hh determine which services are running on pnidX-cnt-bln. \\

\textbf{Solution:} Wir f�hren zun�chst nmap auf dem Host pnid4-cnt-bln aus und �bergeben dabei die Zieladresse des Hosts pnid4-mid-hh, damit wir sehen k�nnen welche Ports ge�ffnet sind bzw. welcher Service auf dem Zielhost gerade aktiv ist. \\

[IMAGE] \\

Wir sehen nun, dass die Ports 21 ftp, 22 ssh, 23 telnet, 80 http und 111 rpcbind offen sind und entscheiden uns via Telnet vom Quellhost pnid4-cnt-bln bei dem Zielhost pnid-mid-hh anzumelden.

[IMAGE] \\  

Die Ausgabe des letzten Logins zeigt uns, dass die Anmeldung erfolgreich war.  


\section*{Exercise 6: Set up a host-to-host VPN using preshared key}
\addcontentsline{toc}{section}{Exercise 6: Set up a host-to-host VPN using preshared key} 

\section*{Exercise 7: Set up a host-to-host VPN using RSA keys}
\addcontentsline{toc}{section}{Exercise 7: Set up a host-to-host VPN using RSA keys}

\section*{Exercise 8: Set up a network-to-network VPN using preshared key}
\addcontentsline{toc}{section}{Exercise 8: Set up a network-to-network VPN using preshared key}

\section*{Exercise 9: Set up a network-to-network VPN using RSA secrets keys}
\addcontentsline{toc}{section}{Exercise 9: Set up a network-to-network VPN using RSA secrets keys}



%\backmatter 

\chapter{Literaturverzeichnis und Quellenverzeichnis}


\end{document}